\documentclass[12pt]{article}
\usepackage{calendar}
\usepackage[landscape, a4paper, margin=1cm]{geometry}
\usepackage{palatino}
\usepackage{datetime}
\begin{document}
	\pagestyle{empty} % Menghapus nomor halaman
	\setlength{\parindent}{0pt} % Mematikan indentasi paragraf
	\StartingDayNumber=2 % Urutan mulai hari, nilai 1 untuk Minggu dan 2 untuk Senin
	
	%------------JUDUL-----------%
	\begin{center}
		\textbf{\LARGE Jadwal Mata Kuliah} \\
		\textsc{\Large Semester 1}
	\end{center}
	
	%------------MULAI HARI---------------%
	\begin{calendar}{\textwidth}
		\day{}{
		\textbf{\formattime{7}{30}{0} - \formattime{9}{10}{0}} \daysep BAHASA INGGRIS \timesep
		\textbf{\formattime{9}{15}{0} - \formattime{11}{0}{0}} \daysep PANCASILA \timesep
		\textbf{\formattime{11}{10}{0} - \formattime{12}{50}{0}} \daysep TRANSFORMASI DIGITAL \timesep
	}

	%------------HARI SELASA--------------%

		\day{}{
		\textbf{\formattime{9}{20}{0} - \formattime{11}{0}{0}} \daysep DASAR-DASAR ILMU SOSIAL \timesep
		\textbf{\formattime{11}{20}{0} - \formattime{13}{0}{0}} \daysep BAHASA INDONESIA \timesep
	}

	%------------HARI RABU----------------%
	
		\day{}{
		\textbf{\formattime{7}{30}{0} - \formattime{9}{10}{0}} \daysep PENDIDIKAN KARAKTER DAN ETIKA \timesep
		\textbf{\formattime{9}{20}{0} - \formattime{11}{0}{0}} \daysep FILSAFAT ILMU \timesep
		\textbf{\formattime{11}{10}{0} - \formattime{12}{50}{0}} \daysep PENDIDIKAN INKLUSI \timesep
	}

	%------------HARI KAMIS---------------%
		
		\day{}{
		\textbf{\formattime{7}{30}{0} - \formattime{9}{10}{0}} \daysep PSIKOLOGI PENDIDIKAN \timesep
		\textbf{\formattime{9}{20}{0} - \formattime{11}{0}{0}} \daysep STATISTIKA \timesep
		\textbf{\formattime{11}{10}{0} - \formattime{12}{50}{0}} \daysep ILMU PENDIDIKAN \timesep
	}

	%------------HARI JUMAT---------------%
		
		\day{}{}
		
	%------------HARI SABTU---------------%
	
		\day{}{}
		
	%------------HARI MINGGU--------------%
	
		\day{}{}

\finishCalendar
\end{calendar}
\end{document}